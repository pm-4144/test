\documentclass{jreport}
\usepackage{multicol}
\usepackage{multirow}
\usepackage{amsmath}
\usepackage{amssymb}
\usepackage{amsfonts}
\usepackage{mathrsfs}
\usepackage{ascmac}
\usepackage{latexsym}
\usepackage{mathtools}

\usepackage[all]{xy}


%\usepackage{layout}
%\setlength{\textheight}{\paperheight}
%\setlength{\topmargin}{-15.4truemm}
%\addtolength{\topmargin}{-\headheight}
%\addtolength{\topmargin}{-\headsep}
%\addtolength{\textheight}{-15truemm}

%\setlength{\textwidth}{\paperwidth}
%\setlength{\oddsidemargin}{-15.4truemm}
%%\setlength{\evensidemargin}{-23.4truemm}
%\addtolength{\textwidth}{-20truemm}

\usepackage{what_day}






\begin{document}


ヘッダファイル
\begin{tabular}{}
int read_cycle_shape(int *cycle_shape, int i_th);

long int cal_hook(int *young_tabular, int m);

long int cal_dim(int *young_tabular, int m);
long int var_cal_dim(int i_th, int m);

long int cal_ord_cc(int *cycle_shape);
long int var_cal_ord_cc(int i_th, int m);

void cal_young_tabular(int *young_tabular);
void cal_young_tabular_sub(int *young_tabular, int m);

void cal_qsort(int *array, int range_size,int order);
void cal_split(int *cycle_shape, int *split);
void cal_cycle_shape(int *cycle_shape);
void var_cal_cycle_shape(int *cycle_shape);

#endif
#ifndef CAL_CHR_H
#define CAL_CHR_H

#include "common.h"

int cal_chr_0(int *phi);	//chi_s
int cal_chr_1(int *phi);	//trivial
int cal_chr_2(int *phi);	//sign
int cal_chr_3(int *phi);	//fix - 1
int cal_chr_4(int *phi);	//chi_3 * sgn
int cal_chr_5(int *phi);	//chi_a
int cal_chr_6(int *phi);	//chi_a * sgn
int cal_chr_7(int *phi);	//chi_s - chi_1 - chi_3
int cal_chr_8(int *phi);	//chi_7 * sgn

void cal_chr_set(int i_th);

void cal_chr_row(int i_th, int j_th);
void cal_chr_col(int i_th, int j_th, long int init);

void cal_chr_init(int i_th);

void cal_chr_ALL(void);




#endif
#ifndef COMMON_H
#define COMMON_H

#include<stdbool.h>
#include<stdio.h>
#include<stdlib.h>
#include<errno.h>
#include<string.h>
#include<math.h>

#define n 4



#if n <= 8
#	define n_prime 8
#	define TERMINAL 22
#	define cycle_shape_f "cycle_shape_8.txt"
#	define split_fmt "%3d"
#	define SPLIT_SIZE 4
#elif n <= 16
#	define n_prime 16
#	define TERMINAL 231
#	define cycle_shape_f "cycle_shape_16.txt"
#	define split_fmt "%4d"
#	define SPLIT_SIZE 5
#elif n <= 32
#	define n_prime 32
#	define TERMINAL 8349
#	define cycle_shape_f "cycle_shape_32.txt"
#	define split_fmt "%5d"
#	define SPLIT_SIZE 6
#elif n <= 64
#	define n_prime 64
#	define TERMINAL 1741630
#	define cycle_shape_f "cycle_shape_64.txt"
#	define split_fmt "%8d"
#	define SPLIT_SIZE 9
#endif

#define POS_SIGN	(3*n_prime/2)
#define POS_FIX		(POS_SIGN + 3)
#define POS_SPLIT	(POS_SIGN + 6)
#define LINE_SIZE_CS	(POS_SPLIT + SPLIT_SIZE)

#define n1 (n_prime/2)
#define n2 (n_prime/2 + 2)

#define num_cc_f 	"num_cc.txt"
#define num_cc_fmt 	"%7d\n"
#define ord_cc_f 	"ord_cc.txt"
#define dim_f 		"dim.txt"


#define MALLOC_ARRAY(number, type) \
	((type *)malloc((number) * sizeof(type)))
#define PP_MALLOC(pp, number, type) \
	do{\
	pp = MALLOC_ARRAY(number, type *);\
	for(int i = 0; i < number; i++)\
	        if ((*(pp + i) = MALLOC_ARRAY(number, type)) == NULL)\
	                exit( EXIT_FAILURE );\
	}while(0)
#define PP_FREE(pp,number) \
	do{\
	for(int i = 0; i < number; i++)\
		free(*(pp + i));\
	free(pp);}while(0)
#define SQRT(sq) ((int) sqrt((double) sq))


//declare factorial
long int fact[n_prime + 1];

//declare array
int num_cc[n_prime];

#define term (*(num_cc + n-1))

long int *ord_cc;
long int *col_range;

int *split;
int *dim;

//declare arrays as character of representation
int *chr;
int *sgn;
int *fix;

//declare a mat as a charcter table
int **chi;

int *row1;
int *row2;
int *col;


int flag;

#endif
#ifndef MAIN_H
#define MAIN_H


#include "mk_file.h"





#endif
#ifndef MK_ARRAY_H
#define MK_ARRAY_H

#include "cal_chr.h"
#include "cal.h"

void mk_array_fact(void);
int mk_array_num_cc(void);

int mk_array_ord_cc(void);
int mk_array_split(void);

//make array as character of representation
int mk_array_char(void);
int mk_array_sgn(void);
int mk_array_fix(void);

//
int mk_array_dim_sub(int m);
int mk_array_dim(void);
int re_array_dim(void);
int mk_array_col_range(void);
void mk_array_search_range(void);


void MALLOC_ALL(void);
void FREE_ALL(void);
void mk_array_ALL(void);


#endif
#ifndef MK_FILE
#define MK_FILE

#include "mk_array.h"

int mk_file_cycle_shape(void);
int mk_file_num_cc(void);
int mk_file_ord_cc(void);
int mk_file_dim(void);

void mk_file_ALL(void);

#endif










\end{document}
